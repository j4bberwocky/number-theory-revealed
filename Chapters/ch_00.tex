% !TEX TS-program = pdflatex
% !TEX root = ../number-theory.tex

%************************************************
%\chapter{Preliminary Chapter on Induction}\label{chp:preliminary}
%************************************************

%\section{Fibonacci numbers and other recurrence sequences}


\begin{solexe}{0.1.1}
    For \( (a) \): the base step is obvious \( F_0 \in \mathbb{N} \); for the induction step,
    we assume that \(F_{n-1}, F_{n-2} \in \mathbb{N} \), but \(\mathbb{N}\) is closed under
    the sum so we have \(F_{n-2} + F_{n-1} \in \mathbb{N} \Rightarrow F_{n} \in \mathbb{N}\).
    \newline
    For \((b)\), remember that:
    \begin{equation*}
        \phi = \frac{1+\sqrt{5}}{2}, 1-\phi=\frac{1-\sqrt{5}}{2} \mbox{ and } \phi^2=\phi+1.
    \end{equation*}

    The base step is trivial. For the induction step we assume:

    \begin{equation*}
        \begin{aligned}
            F_{n-2} = \frac{1}{\sqrt{5}}
            \left(
                \phi^{n-2} - {(1-\phi)}^{n-2}
            \right) \\
            F_{n-1} = \frac{1}{\sqrt{5}}
            \left(
                \phi^{n-1} - {(1-\phi)}^{n-1}
            \right)
        \end{aligned}
    \end{equation*}

    and then

    \begin{equation*}
        \begin{aligned}
            F_{n-2} + F_{n-2} = \frac{1}{\sqrt{5}}
            \left[
                \phi^{n-2}-{(1-\phi)}^{n-2}+\phi\phi^{n-2}-(1-\phi){(1-\phi)}^{n-2}
            \right] \\
            = \frac{1}{\sqrt{5}}
            \left[
                \phi^{n-2}(1+\phi)-{(1-\phi)}^{n-2}(2-\phi)
            \right] \\
            = \frac{1}{\sqrt{5}}
            \left(
                \phi^n - {(1-\phi)}^n
            \right) = F_n,
        \end{aligned}
    \end{equation*}
    Because \( {(1-\phi)}^2=2-\phi \).

\end{solexe}

\begin{solexe}{0.1.2}
    For \( (a) \):
\end{solexe}
